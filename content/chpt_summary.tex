\chapter{Összegzés}

A streaming rendszereket kiszolgáló technológiák fejlődése közel sem ért véget, a QUIC protokoll előretörésével, új modern CDN-funkcionalitásokkal, szerverfarmokban használt alkalmazásspecifikus chipek fejlődésével, illetve hatékonyabb kodekek megjelenésével egyre magasabb minőséget lehet szolgáltatni. Az általam bemutatott rendszer egy egyszerűsített példája a streaming rendszereknek, amely lehetőséget ad arra, hogy megértsük a mögöttes technológiákat és azok működését, betekintést nyújt néhány tipikusan streamingre használt AWS-erőforrás világába, a tervezési technikák palettájába.

A fejlesztés ideje alatt ahogy egyre több komponenst kötöttem a rendszerbe, nőtt a felhőszolgáltatások költsége is. Az dolgozat írásához többször újrateszteltem a futó rendszert, a márciusi összköltsége AWS-ben \$67.98 lett, áprilisban \$73.29, ezekben az összegekben a legjelentősebbek közé tartozott az RDS-adatbázis (\$14--\$17 között), az ALB-példány (\$17--\$20 között) és az ECS-klaszter (\$10--\$11 között) költsége, ezek nem on-demand jellegűek, pár szolgáltatás (Lambda, CloudWatch Logs) pedig a \emph{free tier}, azaz ingyenes átkategóriában tudott maradni. A májusra a Cost Management \$78-t jelzett előre, a hónapban végzett k6 tesztelés próbálgatása miatti (\ref{sec:vod_test}. alfejezet) kb. 100 GB CloudFront-forgalom se emelte jobban.

\section{Továbbfejlesztés lehetőségei}

Architekturális szempontból a webszerver átszervezhető mikroszolgáltatásos alapokra, amennyiben a jövőben a rendszer bővítése, új funkciók bevezetése indokolttá teszi, ezzel is jobban támogatva a már megkezdett event-driven architecture (EDA)\cite{eda} irányvonalat.

Biztonság szempontjából a rendszer továbbfejlesztése érdekében érdemes lenne megfontolni a naplózást kiterjeszteni a CDN és a WAF szintjén ``access logging''-ra is, a VPC-n belül a Flow logok bekapcsolására. Élő, nagy forgalmű rendszerben kihagyhatatlan kötelességünk volna a WAF ACL-ben foglalt szabályokat finomhangolni veszélyes forgalom kiszűrésére. A \ref{sec:vpc}.~alfejezetben említett NAT Gateway és VPC Endpoint megoldások alkalmazása is javallott volna.

A kiszolgálás minőségének javítására érdemes lehet akár a CloudFront-disztribúció cache beállításait is optimalizálás szempontjából újraszemlézni. Felmerülhet egy nagyobb rendszernél a webszerver felé irányított forgalom hálózati terheléséből származó kihívásokra felkészíteni az ALB-példányt is, erre lehet igénybe venni Load balancer Capacity Unit (LCU) foglalást, amely biztosít megfelelő számítási kapacitást a forgalom fogadására. Emellett az ECS-klaszterbe is be lehet vezetni egy automatikus skálázási megoldást, amely a forgalom növekedésével automatikusan új taszkokat indít el az ECS-szolgáltatásban, ezzel biztosítva a megfelelő számítási kapacitást.

A rendszer erős kódmigrálást, infrastruktúra-átalakítást igényelne, amennyiben cél volna a \emph{vendor lock-in}\cite{lockIn} kockázatainak mitigálása. Ehhez javasolt lenne a Kubernetesre való áttérés ECS-ről\cite{k8s}, Ceph\footnote{\url{https://ceph.io/en/}} használata S3 helyett, a Kubernetes alá hozható az adatbázis, illetve a Lambda-függvények is kis mikroszolgáltatások formájában. A MediaConvert, a MediaLive és MediaPackage szolgáltatások kiváltása már bonyolultabb lehet, ebben segíthet az általam korábbi projektben megismert FFMpeg és az NGINX RTMP-modulja\cite{rtmpNginx}, ahogy azt a \ref{sec:elso_lepesek} alfejezetben is említettem.

Nem utolsó sorban a live streaming kiterjesztése még egy feladat, ami várat magára. A rendszer jelenleg nem engedi több élő adás indítását, ezt lehetne kiterjeszteni kódból történő dinamikus MediaLive- és MediaPackage-csatornák felkonfigurálására.
