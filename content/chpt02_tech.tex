\chapter{Technológiai áttekintés}

A videó egy multimédiás eszköz auditív és mozgó vizuális információ tárolására, visszajátszására. Fontos felhasználási területei a média és a szórakoztató ipar. Ebben a fejezetben azon technológiák kerülnek vizsgálatra, amelyek ismerete nélkülözhetetlen a videó streaming megértéséhez, valamint az streaming megvalósításához.

\section{A videó formátumai}

Egy videó tartalmazhat különböző nyelvű hanganyagokat, mozgóképet, és egyéb metaadatokat -- például feliratokat és miniatűr állóképeket -- mind egy fájlban. Ezek közös tárolására konténerformátumokat alkalmazunk, amelyek megadják, az egyes adatfolyamok hogyan, milyen paraméterekkel, kódolással, tömörítéssel kerüljenek tárolásra, és hogyan kerülhetnek majd lejátszásra.

Szokásos összekeverni, de a konténerformátumoktól függetlenül a mozgóképkódolás és a hangkódolás különálló folyamatok. A kódolás egy algoritmus nyomán az adatot tömöríti, hogy a tárolás és a továbbítás hatékonyabb legyen.

\subsection{Konténerformátumok}

A legelterjedtebb és legszélesebb körben támogatott konténerformátum a Moving Picture Experts Group (MPEG) gondozásában specifikált MP4 -- avagy a sztenderdben használt nevén: MPEG-4 Part 14 --, amely az MPEG-4 projekt részeként született 2001-ben.

Ugyancsak az MPEG gondozásában, az MPEG-2 projekt részeként született 1995-ben az MPEG Transport Stream (MPEG-TS) konténerformátum, amely elsősorban a digitális televíziózásban használatos, és így az internetes videó streaming során is. Felbontja a videóadatokat kisebb, fix hosszú adatcsomagokra, ezzel is előkészítve a tulajdonképpeni kis késleltetésű azonnali továbbítására a videóanyagnak a hálózaton keresztül.

További elterjedt konténerformátumok közé tartozik a Matroska Video (MKV), amely a hibatűréséről ismert; Apple vállalat által macOS-re és iOS-re optimalizálva fejlesztett QuickTime Movie (MOV) formátuma; valamint a WebM, egy szabad felhasználású webes kiszolgálásra optimalizálódott formátum, amelyet nagyobb jelentőségű webböngészők mind támogatnak. Régebbi, már kevésbé használt vagy kivezetett formátumok közé tartozik az Audio Video Interleave (AVI) és a Flash Video (FLV).

\subsection{Mozgókép kódolási módszerei}

Az MPEG-4 projekt keretében született az Advanced Video Coding (AVC) -- avagy H.264 -- kódolási szabvány mozgókép kódolására, amely 2004-ban vált elérhetővé. Továbbra is ez az egyik legelterjedtebb szabvány, a videó streamingben használt konténerformátumok is ezt a kódolást alkalmazzák.

Természetesen azóta több új szabvány is megjelent hasonlóan az MPEG projektjei alatt, mint például az 2013-ban megjelent High Efficiency Video Coding (HEVC) -- avagy H.265 --, amely az AVC-től jobb tömörítést és jobb minőséget ígér, de a licencdíjak miatt nem vált annyira elterjedtté, mint az elődje.

Ezen modern problémák kiküszöbölésére terjedt el a VP8 és a VP9 -- a WebM formátumnak tagjaként --, illetve az AV1 szabad felhasználású kódolási szabványok.

A szabványok konkrét megvalósításával (kodekek) nem foglalkozunk részletesebben, de egy szabad felhasználású és nyílt forráskódú megvalósítása a H.264-nek a x264, amelyet például az FFMpeg szoftvercsomag is használ videó kódolására és dekódolására.

\subsection{Hang kódolási módszerei}

Hanganyag kódolására több elterjedt szabványt is Ilyen az MPEG Audio Layer III -- rövid nevén: MP3 -- is 1991-ből, ezt 1997-ben az Advanced Audio Coding (AAC) szabvány váltotta le. Ezen szabványok az MPEG által kerültek kifejlesztésre, valamint mindkettő veszteséges tömörítést alkalmaz, azaz a kódoláson és dekódoláson átesett hanganyag minősége nem lesz azonos az eredeti hanganyaggal.

Egy másik, az előbbieknél jobb minőséget ígérő veszteséges tömörítéssel dolgozó kódolási szabvány Dolby AC-3 -- ismertebb nevén: Dolby Digital.

\section{Videó streaming kiszolgálása}

A médiastreamelés egy olyan folyamat, amely során az médiaadatokat -- mi esetünkben videóadatot -- egy adott hálózati protokoll felett, egy adott konténerformátumban, adott kódolásssal továbbítjuk a végfelhasználók számára. Elsősorban az azonnali elérhetőségre összpontosít, azaz a lejátszásnak a lehető legkisebb késleltetéssel kell megtörténnie, kevésbé fontos a streaming során a minőség megtartása, mint ahogy az fontos lenne teljes médiumok egyben való letöltésekor.

A streaming során az adatfolyamba helyezés előtt a videóadatokat újrakódoljuk, majd kisebb adatcsomagokra -- úgynevezett ``packetekre'' -- bontjuk, és ezeket a csomagokat a hálózaton keresztül továbbítjuk a végfelhasználók felé. Az adatcsomagok önmagukban is értelmezhetőek, és a végfelhasználók lejátszó alkalmazásai képesek az adatcsomagokat a megfelelő sorrendben és időzítéssel lejátszani. A streaming könnyen reagál a lejátszás során ugrálásokra, előre- és visszatekerésre, mivel a videót nem kell teljes egészében letölteni a végfelhasználói eszközre, hanem a feldarabolt videócsomagot a lejátszás során továbbítjuk abban a pillanatban, amikor arra szükség lesz.

\subsection{Az élő közvetítés és video-on-demand különbségei}

Annak megfelelően, hogy az adat egésze mikor áll rendelkezésünkre, kettő fő streaming típust különböztetünk meg: az élő közvetítést (live streaming) és a video-on-demand (VOD) streaminget. A VOD esetében a videóadatokat előre rögzített formában tároljuk, és a végfelhasználók a videóadatokat a saját időbeosztásuknak megfelelően nézhetik meg. Az élő közvetítés esetében a videóadatokat valós időben továbbítjuk a végfelhasználók felé, és a végfelhasználók a videóadatokat a közvetítés során nézhetik meg.

Különbözik mindkettőnél, hogy milyen fizikai infrastrukturális tervezést kell végrehajtani a legjobb kiszolgálás érdekében. Élő adásoknál a késleltetés a középponti kihívás, mivel a videóadatokat a lehető leggyorsabban kell továbbítani a végfelhasználók felé, hogy a közvetítés valóban élőnek tűnjön, ehhez magas számítási teljesítmény szükséges. A VOD esetében a hálózati sávszélességből adódó problémák leküzdése a központi kihívás, mivel a videót a felhasználók sokkal nagyobb közönsége kívánja elérni, azt kiváló minőségben szeretné megtekinteni, ennek ellenére a globális sávszélesség korlátozott. Itt a caching és a tartalomterjesztés optimalizálása a kulcs, ekkor jönnek képbe a Content Delivery Networkök (CDN-ek), azaz a tartalomterjesztő hálózatok.

Üzleti szempontból a bevétel az élő adások során a közbeiktatott reklámokból származik főleg és a pay-per-view rendszerekből, míg a VOD esetében a közbeiktatott reklámokon kívül a felhasználók előfizetési díjából, tranzakcionális egyszeri vásárlásból -- amennyiben a reklámokat kerülni szeretnék.

\subsection{Adaptive Bitrate Streaming}

A streamelést erősen befolyásoló tényező a hálózati feltételek változása, amelyek a videólejátszás minőségét és késleltetését is befolyásolják. Az Adaptive Bitrate Streaming (ABR) technológiával a videólejátszó alkalmazás képes a hálózati feltételek változására reagálni.

Az ABR technológiával a videóadatokat több különböző bitrátával kódoljuk a médiaszerveren, és a lejátszó alkalmazás a hálózati feltételeknek megfelelően választja ki a megfelelő bitrátájú videóadatokat a lejátszásra onnan.

TODO: KÉSZíTS EGY ÁBRAJZATOT AZ ABR MŰKÖDÉSÉRŐL!

\section{Videó streaming hálózati protokolljai}

Videó streamelésére -- másképp fogalmazva: valamely korábban ismertetett formátumban tárolt videó adatfolyamba való illesztésére különböző hálózati protokollok palettája áll rendelkezésünkre.\cite{Wettl04}

Ez a szekció az alkalmazás rétegben alkalmazott protokollok (Layer 7) közül vizsgál meg néhány streaming használatára elterjedt protokollt, megemlítve, milyen szállítási rétegű protokollokkal (Layer 4) tudnak együttműködni.

\subsection{Real-Time Messaging Protocol}

TODO

\subsection{HTTP Live Streaming}

TODO

\subsection{Dynamic Adaptive Streaming over HTTP}

TODO

\subsection{WebRTC}

A források és a streamet figyelők számának kardinalitása szempontjából ez a protokoll a legjobb választás, amennyiben a felhasználási célja az, hogy a felhasználók közvetlenül egymással kommunikálhassanak, és nem szükséges közbeiktatni egy központi médiaszervert. 

TODO: ez bullshitnak hangzik!

\section{Az FFMpeg szoftvercsomag}

TODO

\section{Amazon Web Services}

Bevezető az AWS szolgáltatások világába, röviden bemutatom, hogy épül fel egy AWS account, milyen szolgáltatáscsoportokat ajánl, mivel lehet korlátozni a hozzáférést, az erőforrások akár egymás közötti kommunikációjá irányítani.

\subsection{AWS Elemental}

TODO: MediaConvert

TODO: MediaLive

TODO: MediaPackage

\subsection{Egyéb megoldások videó streamingre AWS-en}

IVS, és saját konténerizált ffmpeg szerver miért nem.

\subsection{Amazon CloudFront}

Korábbi szekcióban már említettük a CDN-eket, mint a video-on-demand alapú streaming szolgáltatások egyik kulcsfontosságú elemeit. Az Amazon CloudFront egy globális CDN, amely lehetővé teszi a felhasználók számára, hogy a tartalmat hozzájuk közelebbi szervereken tárolt cache-ből töltsék le, ezáltal csökkentve a késleltetést, csökkentve a központi szerverek terhelését, és növelve a letöltési sebességet.

\subsection{Amazon ECS}

TODO

\subsection{Amazon RDS}

TODO

\subsection{Orkesztrációs szolgáltatások}

Lambda és EventBridge, IAM role-ok, hálózati infra.

\section{A webes komponensek technológiái}

Ebben a részben a konkrét AWS-beli PaaS és FaaS szolgáltatásokon futó több rétegű appok választott technológiáit mutatom be: Node.js, React, Prisma, Postgres.

\subsection{TypeScript és JavaScript nyelvek}

Miért választottam a TypeScriptet, miért jobb a JavaScriptnél, felhasználása a JavaScriptnek a Lambda és Cloudfront Function-ökben.

\subsection{Node.js ökoszisztéma}

Node.js futtatókörnyezet, NestJS keretrendszer, Prisma ORM, AWS SDK.

\subsection{React}

TODO: Mire jó a React: Statikus oldalak generálása (SSG), egyszerűbb JavaScript egy helyen kezelése, gyors build toolok, erre ott a Vite.js.

TODO: Sok könnyen integrálható könyvtár, amelyek segítségével gyorsan és hatékonyan lehet webes felületeket fejleszteni: React Hook Forms, React Query.
