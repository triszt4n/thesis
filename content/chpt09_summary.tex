\chapter{Összegzés}

TODO: Tanulságok, a rendszer működésének és fejlesztési élmények értékelése. Média streaming jövője saját meglátások szerint.

\section{A felhasznált erőforrások költségei}

TODO: Költségek eloszlása.

\section{Népszerű szolgáltatók metrikái}

A média streaming szolgáltatások fejlesztése és üzemeltetése során a mérési eredmények alapján lehet a legjobban optimalizálni a rendszert, kiszámolni a költségeket, és a felhasználói élményt folyamatosan javítani. A fejlesztett alkalmazás túl kicsi volt és felhőbéli fejlesztéssel való kísérletezés volt a célja, így nem lehetett valós mérési eredményeket gyűjteni. Az alábbiakban bemutatok néhány példát a legnagyobb szolgáltatók által használt metrikákra, amelyek segíthetnek a jövőbeli fejlesztések során.

TODO: Keresni kell cikkeket Netflix blogban vagy valahol arról, a népszerű szolgáltatók milyen SLA-val, milyen statisztikával dolgoznak.

TODO: ebben talán van valami https://openconnect.netflix.com/Open-Connect-Briefing-Paper.pdf 


\section{Továbbfejlesztés lehetőségei}

TODO: Min lehetne javítani szoftverügyileg, mikroszolgáltatásos architektúra stb.

TODO: Hogy lenne ez még profibb munka DevSecOps szempontból
Logging minden szinten: WAF, Route53, Cloudfront access logs, ALB
Athenával szűrni
Dashboardok a bejövő forgalomra
WAF fine-tuning, better botmgmt
ALB-re reserved LCU plan, ha folytonos lenne a forgalom
védettebb ECS cluster: privát subnet
NAT gatewayjel
Forgalom csekkolása: VPC Flow logs
Most hogy vannak a cache behaviourok
Ki vannak kapcsolva
Videók egyedi hashelt pathon vannak, akár 1 évre lehetne őket cachelni
Javascriptek 5min-nel mehetnének
VPC Origin hátrányait ismertetni
cross region és cross account nem műxik
nem támogatja a grpc-t és a websocketet, ez eléggé bekorlátol (egyelőre)

\subsection{Vendor lock-in jelensége}

TODO: Miként befolyásolja egy üzlet működését a vendor lock-in, és hogyan lehet ezt kezelni. Miképp lehetne a jövőben a vendor lock-in-t csökkenteni ebben a rendszerben. S3 helyett Ceph, konténert kiemelni, akár K8s-re felkészíteni, hogy ne ECS-től függjön.
