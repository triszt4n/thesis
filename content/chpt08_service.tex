\chapter{Rétegeken átívelő szolgáltatások implementációja}

Ebben a fejezetben kerül kifejtésre az implementációja olyan szolgáltatásainak az alkalmazásnak, amelyek a kliens és szerver rétegén átívelnek, és külön-külön nem lenne érdemes bemutatni őket.

\section{Single Sign-On (SSO) integrációja}

A videók feltöltésére azok jogosultak, akik a stúdióba be tudnak lépni a \verb|/studio| aloldalon, a bejelentkeztetéshez pedig a kari hallgatói közösségünk által nyílt forráskóddal fejlesztett AuthSCH nevű Single Sign-On (SSO) rendszert integráltam be a weboldalba,  ennek az esszenciális tokenkezelési implementációs részeit a backendre bíztam, a kliensoldal csupán a megfelelő útvonalakra irányításért felel.

A \ref{lst:authCtx}. kódrészlet mutatja be a React-alkalmazásban használt \verb|AuthCtx| nevű kontextust, amely JSON Web Tokeneket (JWT) használ a bejelentkeztetés során a felhasználói adatok biztonságos és állapotmentes átpasszolására. A \verb|useMe| nevű hook az AuthSCH SSO API-ját használja a bejelentkezett felhasználó adatainak lekérdezésére, amelyet a \verb|/api/me| végponton keresztül érhetünk el. A \verb|AuthProvider| komponens biztosítja a kontextust az alkalmazás többi részének, és kezeli a bejelentkezést és kijelentkezést.

TODO: befejezni.

\begin{minipage}{0.92\textwidth}
  \begin{lstlisting}[
  caption=src/hooks/auth-context.tsx fájl tartalma.,
  label=lst:authCtx,
  style=js,
  basicstyle=\fontsize{10}{12}\ttfamily
]
import { createContext, PropsWithChildren } from "react"
import { useMe } from "./use-me.hook"

type AuthCtxType = {
  authenticated: boolean
  isLoading: boolean
  login: () => void
  logout: () => void
}
export const AuthCtx = createContext<AuthCtxType | undefined>(undefined)

export function AuthProvider({ children }: PropsWithChildren) {
  const { data, isLoading } = useMe()
  const onLogin = async () => {
    window.location.href = import.meta.env.VITE_BACKEND_URL+"/auth/login"
  }
  const onLogout = async () => {
    window.location.href = import.meta.env.VITE_BACKEND_URL+"/auth/logout"
  }
  const value = {
    authenticated: !!data,
    isLoading,
    login: onLogin,
    logout: onLogout,
  }
  return <AuthCtx.Provider value={value}>{children}</AuthCtx.Provider>
}
\end{lstlisting}
\end{minipage}

TODO: Szerveroldali controller.

\begin{minipage}{0.92\textwidth}
  \begin{lstlisting}[
    caption=Az AuthController osztály fontos függvényei.,
    label=lst:authController,
    style=js,
    basicstyle=\fontsize{10}{12}\ttfamily
  ]
@UseGuards(AuthSchGuard)
@Get("login")
@ApiFoundResponse({
  description: "Redirects to the AuthSch login page.",
})
login() {}

@Get("callback")
@UseGuards(AuthSchGuard)
@ApiFoundResponse({
  description: "Redirects to the frontend and sets cookie with JWT.",
})
@ApiQuery({ name: "code", required: true })
oauthRedirect(@CurrentUser() user: UserDto, @Res() res: Response): void {
  const jwt = this.authService.login(user)
  res.cookie("jwt", jwt, {
    httpOnly: true,
    secure: true,
    domain: process.env.NODE_ENV === "production" ? getHostFromUrl(process.env.FRONTEND_CALLBACK) : undefined,
    maxAge: 1000 * 60 * 60 * 24 * 7, // 7 days
  })
  res.redirect(302, process.env.FRONTEND_CALLBACK + "?authenticated=true")
}
\end{lstlisting}
\end{minipage}
