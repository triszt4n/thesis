\chapter{Rétegeken átívelő szolgáltatások implementációja}

Ebben a fejezetben kerül kifejtésre az implementációja két szolgáltatásnak az rendszerben, amelyek a kliens (React) és a szerver (Node.js) rétegén átívelnek: a Single Sign-On (SSO) integrációja és a videófeltöltés folyamata.

\section{Single Sign-On integrációja}

A videók feltöltésére azok jogosultak, akik a stúdióba be tudnak lépni a \verb|/studio| aloldalon, a bejelentkeztetéshez pedig a kari hallgatói közösségünk által nyílt forráskóddal fejlesztett AuthSCH nevű SSO-rendszert integráltam be a weboldalba, ennek az esszenciális tokenkezelési implementációs részeit a backendre bíztam, a kliensoldal csupán a megfelelő útvonalakra irányításért felel.

\Az+\ref{lst:authCtx}. kódrészlet mutatja be a React-alkalmazásban használt \verb|AuthCtx| nevű kontextust, amely JSON Web Tokeneket\footnote{\url{https://jwt.io/}} (JWT)\cite{jwt} használ a bejelentkeztetés során a felhasználói adatok biztonságos és állapotmentes átpasszolására. A \verb|useMe| nevű hook a backend \verb|/api/auth/me| végpontjáról kéri le a bejelentkezett felhasználó adatait, a végpont mögötti logika csupán annyiból áll, hogy kibontja a kódolt JWT-t és visszaadja abból a releváns profiladatokat. Az \verb|AuthProvider| nevű React-függvénykomponens biztosítja a kontextust az alkalmazás többi részének, a \verb|useAuth| hookon keresztül lehet indítani a többi komponensekben bejelentkezést és kijelentkezést (\verb|login| és \verb|logout| függvények), lehet megtudni, van-e bejelentkezett felhasználó a rendszerben (\verb|authenticated| boolean változó).

\begin{minipage}{0.92\textwidth}
  \begin{lstlisting}[
  caption=auth-context.tsx fájl tartalma.,
  label=lst:authCtx,
  style=js,
  basicstyle=\fontsize{10}{12}\ttfamily
]
import { createContext, PropsWithChildren } from "react"
import { useMe } from "./use-me.hook"

type AuthCtxType = {
  authenticated: boolean
  isLoading: boolean
  login: () => void
  logout: () => void
}
export const AuthCtx = createContext<AuthCtxType | undefined>(undefined)

export function AuthProvider({ children }: PropsWithChildren) {
  const { data, isLoading } = useMe()
  const onLogin = async () => {
    window.location.href = import.meta.env.VITE_BACKEND_URL+"/auth/login"
  }
  const onLogout = async () => {
    window.location.href = import.meta.env.VITE_BACKEND_URL+"/auth/logout"
  }
  const value = {
    authenticated: !!data,
    isLoading,
    login: onLogin,
    logout: onLogout,
  }
  return <AuthCtx.Provider value={value}>{children}</AuthCtx.Provider>
}
export function useAuth() {
  const context = useContext(AuthContext)
  if (context === undefined) {
    throw new Error("useAuth must be used within an AuthProvider")
  }
  return context
}
\end{lstlisting}
\end{minipage}

A szerver oldalán egy \verb|AuthModule| névre hallgató NestJS-modul került létrehozásra, amely a \verb|AuthController| és \verb|AuthService| osztályokat tartalmazza. Az \verb|AuthController| osztályban kerülnek definiálásra a forgalmat lehallgató HTTP-végpontok/függvények, míg az \verb|AuthService| osztály elkülöníti a konkrét üzleti logikát. \Az+\ref{lst:authController}. kódrészlet bemutat két fontos HTTP-végpontot, a \verb|/api/auth/login| és a \verb|/api/auth/callback| útvonalakon GET metódusra hallgatókat sorrendben.

\begin{minipage}{0.92\textwidth}
  \begin{lstlisting}[
    caption=Az AuthController osztály fontos függvényei.,
    label=lst:authController,
    style=js,
    basicstyle=\fontsize{10}{12}\ttfamily
  ]
@UseGuards(AuthSchGuard)
@Get("login")
@ApiFoundResponse({
  description: "Redirects to the AuthSch login page.",
})
login() {}

@Get("callback")
@UseGuards(AuthSchGuard)
@ApiFoundResponse({
  description: "Redirects to the frontend and sets cookie with JWT.",
})
@ApiQuery({ name: "code", required: true })
oauthRedirect(@CurrentUser() user: UserDto, @Res() res: Response): void {
  const jwt = this.authService.login(user)
  res.cookie("jwt", jwt, {
    httpOnly: true,
    secure: true,
    domain: process.env.NODE_ENV === "production" ? getHostFromUrl(process.env.FRONTEND_CALLBACK) : undefined,
    maxAge: 1000 * 60 * 60 * 24 * 7, // 7 days
  })
  res.redirect(302, process.env.FRONTEND_CALLBACK + "?authenticated=true")
}
\end{lstlisting}
\end{minipage}

A \verb|AuthSchGuard| egy NestJS-dekorátor, amely fel kell kerüljön a \verb|oauthRedirect| és \verb|login| függvényekre. E mögött a dekorátor mögött egy Passport.js-stratégia\cite{passport} van, amely megvalósítja az AuthSCH-val való OAuth-alapú beazonosítási logikát, az OAuth-token megújítását. Ehhez kell beállítsuk környezeti változóként a kapott OAuth-kliens azonosítóját és titkos kulcsát, ezen értékek forrásáról korábban \az+\ref{sec:envvars}. alfejezet beszél.

A \verb|login| függvényt maga az \verb|AuthSchGuard| dekorátor kiváltja, a függvény törzsébe nem kell semmit se írni, a mögöttes logika a felhasználót átirányítja az AuthSCH bejelentkezési oldalára. Az \verb|oauthRedirect| függvény a sikeres bejelentkezés után visszairányítja a~felhasználót a frontendre, a kérésben a \verb|jwt| nevű HTTP-only biztonságos sütit beállítja\cite{ietf-oauth-security-topics-13}, amely tartalmazza a bejelentkezett felhasználóra jellemző JWT-t 7 napos lejárati idővel. A \verb|@CurrentUser()| dekorátor a bejelentkezett felhasználó adatait injektálja a megadott \verb|user| paraméterbe -- ezt az \verb|AuthSchGuard| oldja meg a háttérben, kiolvassa a~kapott autorizációs kódot, amit az AuthSCH-tól kapott. Az \verb|AuthService| osztály \verb|login| függvénye a bejelentkezett AuthSCH-s felhasználó adatait kapja meg, amelyből a JWT-t generálja, ezt \az+\ref{lst:authService}. kódrészlet mutatja be. A \verb|createOrUpdateUser| függvény a Prisma ORM segítségével megkeresi a felhasználót az adatbázisban, ha nem találja, akkor létrehozza azt, ha megtalálja, akkor frissíti a felhasználó adatait, ez a függvény kerül felhasználásra az \verb|AuthSchGuard| mögötti stratégiában is a bejelentkeztetés validációs ``mellékhatásaként'', hogy a saját adatbázisunkban is létrejöjjön a felhasználó.

\begin{minipage}{0.92\textwidth}
  \begin{lstlisting}[
    caption=Az AuthService osztály fontos függvényei.,
    label=lst:authService,
    style=js,
    basicstyle=\fontsize{10}{12}\ttfamily
  ]
login(user: object): string {
  return this.jwtService.sign(user, {
    secret: this.configService.get<string>("JWT_SECRET"),
    expiresIn: "7 days",
  })
}
async createOrUpdateUser(prof: AuthSchProfile): Promise<UserDto> {
  const gravatarUrl = this.getGravatarUrl(prof.email, 200)
  return this.prisma.user.upsert({
    where: { authSchId: prof.authSchId },
    update: {
      fullName: prof.fullName,
      firstName: prof.firstName,
      email: prof.email,
      imageUrl: gravatarUrl,
    },
    create: {
      authSchId: prof.authSchId,
      fullName: prof.fullName,
      firstName: prof.firstName,
      email: prof.email,
      imageUrl: gravatarUrl,
    },
  })
}
\end{lstlisting}
\end{minipage}

Lefejlesztésre került egy másik \emph{guard} típusú NestJS-dekorátor, ez a \verb|JwtGuard|, amelyet minden más a szerveren található végpontra be kell vezetni, amennyiben az bejelentkezett felhasználót igényel, ez a dekorátor ellenőrzi a \verb|jwt| süti jelenlétét, a felhasználót beazonosítja. Az ilyen függvényekben is, ha a paraméterre egy \verb|@CurrentUser()| dekorátort helyezünk, akkor a felhasználó adatai automatikusan be lesznek injektálva abba a~paraméterbe.

\section{Videófeltöltés folyamata}

A videófeltöltés lehetőségét helyileg a stúdió aloldalán videóprojekt létrehozása után a~szerkesztőmódban találhatjuk meg, a feltöltésre egy ``dropzone'' került implementálásra, amely egy olyan input mező, amely képes lekezelni ``drag and drop'' (magyarul \emph{fogd és vidd}) módszerrel ráejtett fájlokat, azt felküldeni a szervernek. Ennek kódjából mutat be részletet \az+\ref{lst:dropzone}. kódrészlet.

\begin{minipage}{0.92\textwidth}
  \begin{lstlisting}[
    caption=Részlet a Dropzone függvénykomponensből.,
    label=lst:dropzone,
    style=js,
    basicstyle=\fontsize{10}{12}\ttfamily
  ]
export const Dropzone = ({ onChange, ...props }) => {
  const fileInputRef = useRef<HTMLInputElement | null>(null)
  // ...
  const handleDrop = (e) => {
    e.preventDefault()
    e.stopPropagation()
    const { files } = e.dataTransfer
    handleFiles(files)
  }
  const handleFileInputChange = (e) => {
    const { files } = e.target
    if (files) handleFiles(files)
  }
  const handleFiles = (files: FileList) => {
    onChange(() => Array.from(files))
    // ...
  }
  const handleButtonClick = () => {
    fileInputRef.current?.click()
  }
  return (
    <Card ... onClick={handleButtonClick}>
      <CardContent ... onDragOver={handleDragOver} onDrop={handleDrop}>
        <div className="...">
          <span className="font-medium">Húzz egy fájlt ide</span>
          <Button variant="ghost" size="sm" className="...">
            vagy kattints ide
          </Button>
          <input ref={fileInputRef} type="file" onChange={handleFileInputChange} className="hidden" multiple />
        </div>
      </CardContent>
    </Card>
  )
}
\end{lstlisting}
\end{minipage}

A \verb|Dropzone| nevű React-függvénykomponensben maga az input mező egy rejtett HTML-elem, amely a \verb|fileInputRef| referencián keresztül érhető el. A fájlok feltöltéséért a~\verb|handleFiles| függvény felelős, amely a kiválasztott fájlokat egy tömbbé alakítja, és kihív a paraméterben átadott \verb|onChange| függvényre a szülőkomponensben. A konkrét szerver felé kommunikálást a szülőkomponens végzi.

Feltöltés végeztével a szerkesztői felületen megjelenik a ``Videó megtekintése'' gomb, amelyre kattintva elnavigál a weboldal egy új aloldalra, ahol a videólejátszó is implementálásra került. A videólejátszó komponens a \verb|VideoPlayer| nevű React-függvénykomponens, amely a \verb|hls.js| nevű könyvtárat használja a stream letöltés kezelésére. \Az+\ref{lst:videoPlayer}. kódrészlet mutatja be a könyvtár használatát, a komponens implementációját.

\begin{minipage}{0.92\textwidth}
  \begin{lstlisting}[
    caption=A VideoPlayer függvénykomponens kódja.,
    label=lst:videoPlayer,
    style=js,
    basicstyle=\fontsize{10}{12}\ttfamily
  ]
import Hls from "hls.js"
import { useEffect } from "react"

export const VideoPlayer = ({ src, width, height }) => {
  const ref = React.useRef<HTMLVideoElement>(null)
  useEffect(() => {
    if (ref.current && Hls.isSupported()) {
      const hls = new Hls()
      hls.loadSource(src)
      hls.attachMedia(ref.current)
    }
  }, [src])
  return (
    <video width={width} height={height} controls ref={ref}>
      <source src={src} type="video/mp4" />
      Your browser does not support the video tag.
    </video>
  )
}
\end{lstlisting}
\end{minipage}

A szerveroldalon a feltöltésre kerülő nyers videók egy arra kijelölt S3-vödörbe kerülnek továbbításra. A korábban ismertetett (\ref{sec:nodejs}. alfejezet) videókért felelős NestJS-modulbeli \verb|VideoController| és \verb|VideoService| osztályokból emeli ki rendre \az+\ref{lst:videoController}. kódrészlet és \az+\ref{lst:videoService} kódrészlet a fontosabb részeket.

\begin{minipage}{0.92\textwidth}
  \begin{lstlisting}[
    caption=Videófeltöltés kezelése a VideoController osztályban.,
    label=lst:videoController,
    style=js,
    basicstyle=\fontsize{10}{12}\ttfamily
  ]
@Post(":id/upload")
@UseGuards(JwtGuard)
@UseInterceptors(FileInterceptor("file"))
async upload(
  @Param("id") id: string,
  @UploadedFile(new ParseFilePipe()) file: Express.Multer.File
) {
  const { originalname, buffer } = file
  const res = await this.videoService.upload(id, originalname, buffer)
  return await this.videoService.afterUpload(id, res.fileName)
}
\end{lstlisting}
\end{minipage}

A feltöltéshez szükséges fájlok a kliensoldalon kerülnek kiválasztásra, és a POST metódusú \verb|/api/videos/:id/upload| végponton keresztül kerülnek feltöltésre. Ezt a végpontot hallgatja le a \verb|VideoController| osztálybéli \verb|upload| függvény is. A függvény dekorátorai közül a \verb|@Post| jelöli azt, amiképp hallgat a kérésekre (POST metódussal és a jelölt útvonalon). A \verb|@UseGuards| dekorátor az autentikáció szükségességét biztosítja a kérések végbemeneteléhez, amelyet a mi esetünkben a \verb|JwtGuard| osztály valósít meg. A \verb|@UseInterceptors| dekorátor pedig egy interceptort állít be a függvény elé, a fájlok feltöltését segíti elő, amelyet a \verb|FileInterceptor| osztály valósít meg. Ez az osztály a NestJS alatt működő Express keretrendszerben működő \emph{Multer} nevű Node.js-middleware-t hasznosítja, amellyel így tehát képes az osztály mögötti logika a HTTP-alapú bájtfolyamot hatékonyan feldolgozni/parse-olni, és elrejteni előlünk a sok-sok kihívását egy fájlfeltöltésnek. A függvény paramétereiben a \verb|file| paraméter a \verb|@UploadedFile| dekorátoron keresztül válik a feltöltött fájl reprezentációjává, típusa is a \verb|File| interfész lesz.

A \verb|VideoService|-ben működő \verb|upload| függvény felé lesz a kapott fájl bufferje továbbítva, végül a service függvény eredményéből született fájlnévvel kerül az adatbázisba is lementésre az entitásként a videó is a \verb|afterUpload| függvény által.

A \verb|VideoService| az AWS SDK S3 API-ját használja a fájlok vödörbe való feltöltésére. Ehhez létrehoztam tagváltozóként a \verb|s3Client|-et. Az \verb|upload| függvény először átalakítja a fájlnevet, megtartja a kiterjesztést, viszont a dátumot szerkeszti bele a konkrét névbe. Majd pedig készít egy \verb|PutObjectCommand| típusú objektumot, amellyel a megfelelő helyre és névvel kerül feltöltésre az \verb|s3Client|-en keresztül a fájl.

Az ezután történő S3-ba való feltöltésből származó események kezeléséről ír a korábbi \ref{sec:mediaConvert}. alfejezet.

\begin{minipage}{0.92\textwidth}
  \begin{lstlisting}[
    caption=Videófeltöltés függvénye a VideoService osztályban.,
    label=lst:videoService,
    style=js,
    basicstyle=\fontsize{10}{12}\ttfamily
  ]
private readonly s3Client = new S3Client({
  region: this.configService.getOrThrow("AWS_S3_REGION"),
})
async upload(id: string, fileName: string, file: Buffer) {
  const ext = fileName.split(".").slice(-1)[0]
  const baseName = new Date()
    .toISOString().slice(0, 19).replaceAll(":", "-")

  const response = await this.s3Client.send(
    new PutObjectCommand({
      Bucket: this.configService.getOrThrow("AWS_S3_UPLOADED_BUCKET"),
      Key: `${id}/${baseName}.${ext}`,
      Body: file,
    })
  )
  return { ...response, fileName: `${baseName}.${ext}` }
}
\end{lstlisting}
\end{minipage}
