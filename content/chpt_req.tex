\chapter{Követelmények}

Egy nagyszabású, YouTube- vagy Netflix-szintű webes videó streaming szolgáltatás megvalósítása rendkívül összetett feladat, amely köré így kiterjedt technikai és üzleti követelményrendszert tudunk kialakítani. Egy ilyen rendszernek kezdetben biztosítania kell az~alapvető funkciókat, amelyet hétköznapi webalkalmazások is megvalósítanak, mint például a videók lejátszása során felmerülő interakciók kezelése. Azonban ahogy a platformunk népszerűsége növekedhet, újabb és újabb infrastrukturális igények merülhetnek fel: ezek teszik ki a nem funkcionális követelményeket, amik kiterjednek például a tartalomterjesztés minőségére, a biztonsági felkészültségére a webalkalmazásnak.

A következőkben részletezem azokat a követelményeket, elvárásokat, amelyeket szem előtt tartottam a streaming szolgáltatás megtervezésekor és megvalósításakor.

\section{Funkcionális követelmények}

A streaming szolgáltatást alapvetően egy központosított kliensoldali webes felületen keresztül lehet elérni, azon keresztül tudják adminisztrátorok kezelni a tartalmat. Ennek a~közvetlen frontoldali webes felületnek és a szolgáltatásainak a követelményeit érdemesnek tartottam csoportokba szedni:

\begin{itemize}
  \item Felhasználókezelésre vonatkozó funkciócsoport
        \begin{itemize}
          \item Habár a videók megtekintéséhez nincs szükség felhasználókezelésre és bejelentkezésre, viszont a videók kezeléséhez szükséges megvalósítani a felhasználók azonosítását és jogosultságkezelését.
          \item A felhasználók AuthSCH\footnote{\url{https://vik.wiki/AuthSCH}} segítségével tudnak bejelentkezni a rendszer ``stúdió'' nevezetű aloldalára, ahol a rendszer azonosítja őket AuthSCH-s profiljuk alapján, automatikusan jön létre felhasználói fiókjuk, regisztrációra külön nincs szükség.
          \item A felhasználók közötti különbséget a rendszerben adminisztrátorok és normál felhasználók között teszem meg, az előbbieknek jogosultságuk van a videók feltöltésére és kezelésére, az utóbbiaknak csupán bejelentkezésre későbbi adminisztrátorrá válásra, amennyiben másik adminisztrátor úgy dönt.
        \end{itemize}

  \item Videóprojektek kezelésére vonatkozó funkciócsoport
        \begin{itemize}
          \item Az adminisztrátorok a ``stúdió'' aloldalon tudnak új videóprojekteket létrehozni, amelyekhez a rendszer automatikusan generál egy egyedi azonosítót.
          \item Egy videóprojekt létrehozása után tudunk a videókhoz tartozó metaadatokat adni, azokon módosítani, mint például a cím, leírás, borítókép, kategória, résztvevő stábtagok.
          \item A videóprojektben lehetőség van egy darab MP4 konténerformátumú videó feltöltésére.
          \item Van lehetőség a videóprojekt teljes törlésére.
          \item A feltöltés után a felhasználói felület visszajelzést kell adjon arról, hol tart a~videó feltöltési folyamata, illetve a videókonvertálás folyamata a hálózati adatfolyamra való felkészítéshez.
          \item A ``stúdión'' kívüli oldalon a videóprojektek listázva vannak, ahol a nézők megtekinthetik a videóprojektek konkrét videóit stream formájában.
        \end{itemize}

  \item Élő közvetítés kezelésére vonatkozó funkciócsoport
        \begin{itemize}
          \item Az oldal kezdetlegesen csupán egy élő közvetítés adását támogatja, az adminisztrátor a ``stúdió'' aloldalon tudja ezt az egyetlen élő közvetítést indítani.
          \item A rendszer biztosítja, hogy fogadja egy külső forrásból (pl.: OBS Studio alkalmazásból) a felstreamelt videót a felhőn át és azt továbbítja a nézőknek.
          \item Az oldalon kell, legyen egy útvonal a nézők számára, ahol a közvetítés élőben megtekinthető.
        \end{itemize}
\end{itemize}

\section{Nem funkcionális követelmények}

A rendszerrel szemben támasztott nem funkcionális követelmények megállapításakor igyekeztem olyan dolgokra a hangsúlyt tenni, amelyek inkább számomra jelentenek kihívást, mivel nem terveztem a webalkalmazást úgy elkészíteni, hogy az valódi használatra készüljön -- azaz valódi haszna legyen és legyenek élő felhasználói a nagyvilágból, csupán a~kísérletnek volt része. Ezeket az elvárásokat a következőkben állapítottam meg:

\begin{itemize}
  \item Biztonság
  \begin{itemize}
    \item A megoldás kihasználja az AWS-felhő adta biztonsági lehetőségeket mind a hálózati struktúra kialakításakor, a webalkalmazás védelmezésére és a biztonságos kommunikáció (pl. HTTPS) kialakítására.
    \item A webalkalmazásban a felhasználók autentikációját és autorizációját a JWT`|tokenek segítségével oldom meg.
  \end{itemize}
  \item Hordozhatóság és könnyű karbantarthatóság
   \begin{itemize}
    \item Konténerizálom a szerveralkalmazást, hogy könnyen lehessen telepíteni és futtatni, egy egységként lehessen kezelni.
    \item A konténer és a buildelendő kódok automatikusan fordulnak és települnek a~CI/CD-folyamat részeként.
    \item A frissítések könnyedén kezelhetőek, erre alkalmazásra kerül konténerképeket tároló Docker Registry is.
    \item Eseményvezérelt architektúra az alkalmazás egyes folyamatainak megvalósítására és a szerveralkalmazás köré, hogy könnyen kiterjeszthető lehessen új funkcionalitásokkal.
    \item Az infrastruktúra kód formájában (Infrastructure as Code, IaC) is dokumentált, hogy könnyen lehessen újraépíteni a rendszert.
  \end{itemize}
  \item Elaszticitás
  \begin{itemize}
    \item Alkalmazásra kerül olyan futtatókörnyezet, amelyben könnyedén lehet az erőforrásokat növelni és csökkenteni, hogy a rendszer mindig a szükséges kapacitással rendelkezzen.
    \item Alkalmazásra kerül a CDN használata, hogy a videók gyorsan és megbízhatóan érhetőek legyenek el a nézők számára.
  \end{itemize}
  \item Költséghatékonyság
  \begin{itemize}
    \item Olyan szolgáltatások használata, amelyek csak a valóban szükséges erőforrásokat használják fel, és csak akkor, amikor azokra szükség van.
    \item A szolgáltatásokat úgy tervezem meg, hogy a lehető legolcsóbban lehessen őket üzemeltetni a kísérletezés során.
  \end{itemize}
\end{itemize}
