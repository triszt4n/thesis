\chapter{Kliens közeli komponensek implementációja}

\section{A CDN és a hozzácsatolt erőforrások}

A CDN és az originjeinek tárgyalása, melyik origin mire való. VPC Origin tárgyalása, annak haszna. A WAF szerepe, a felkapcsolt cert és WAF web ACL szerepe.

\section{A statikus weboldal}

Az S3 bucketee. A statikus website kiszolgálásának módja.

\subsection{A React alkalmazás fejlesztése}

Fejlesztés részletei. Nem annyira lényeges most ebben a dolgozatban, de néhány szép kihívást jelentő kidolgozott form és egyebeket be lehet itt mutatni.

\subsection{A weboldal telepítésének CI/CD folyamata}

GitHub Actions a smoke tesztekre, workflowk, a deploymentek. Értsd itt: S3 telepítés.

\section{Média erőforrások objektumtárolói}

Hogy lettek felkonfigurálva és miért az egyes S3 bucket-ok (bucket policy, CORS policy).

\section{Elemental MediaLive és MediaPackage a live streamingben}

Az Elemental stack részeinek felkonfigurálása, a MediaLive channel és a MediaPackage channel felépítése, a MediaPackage endpoint konfigurálása. Miként kerül kiszolgálásra, melyiket mire használom. OBS bekötésének módja.
