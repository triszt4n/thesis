\chapter{\bevezetes}

\section{A téma ismertetése}

A média és szórakoztatás egy óriási szeletét tölti ki az internet teljes forgalmának. A videó streaming szolgáltatók, mint például a Netflix, a Twitch, vagy a YouTube, több százmilliós -- vagy akár milliárdos -- aktív felhasználói bázissal rendelkeznek, az globálisan kiterjedt és folyamatos forgalom kiszolgálására a világ legnagyobb szerverparkjait üzemeltetik. Kiterjed a felhasználásuk az élet minden területére: az oktatásra, a szórakozásra, a mindennapi és munkahelyi kommunikációra, a hírek és információk terjesztésére, a kulturális és művészeti élmények megosztására, össze tud kötni embereket a világ minden tájáról.

A média streaming szolgáltatások fejlesztése és fenntartása komoly kihívások elé állítja a tervezőmérnököket, és a szakma legjobbjai a világ minden tájáról dolgoznak azon, hogy a felhasználói élményt folyamatosan javítsák, és a szolgáltatásokat a lehető legnagyobb számú felhasználó számára elérhetővé tegyék.

\section{A témaválasztás indoklása}

Véngponttól végpontig tartó média streaming szolgáltatásoknak a fejlesztése és üzemeltetése számos komplex, informatikai területeket áthidaló kihívásokat hordoz magában. Ki kell tudni alakítani egy fenntartható, globális kiszolgálásra optimalizált és biztonságos hálózati infrastruktúrát. Folyamatosan kell tervezni a skálázhatóság biztosításával növekvő felhasználói bázissal. Ki kell tudni használni a legfrissebb hálózati protokollok adta lehetőségeket. Mérni kell a metrikákat a kitűnő felhasználói élmény biztosítása érdekében. Ki kell tudni szolgálni termérdekféle végfelhasználói hardvert -- például mobiltelefonok, különféle böngészők, AR- és VR-eszközök. A szabványok területén is tájékozottnak kell lennie a mérnököknek.

Már csak a kísérletezéssel felszedhető ismeretek is a piacon óriási előnyt jelentenek az ilyen irányba elköteleződő szakembereknek. Ezen indokok nyomán választottam magam is ezt a témát további vizsgálatra, eredményeim osztom meg jelen diplomamunkában.

\section{Az első lépések}\label{sec:elso_lepesek}

A téma bejárásának megkezdéseképp az Önálló laboratórium 2 című tárgy keretében egy olyan full-stack webes rendszer tervezését és implementációját vállaltam, amely lokálisan is futtatható szabad szoftvereket alkalmaz egy média streaming szolgáltatás alapjainak lefektetésére. Ennek köszönhetően megismerkedtem az FFMpeg szoftvercsomag videókonvertálási lehetőségeivel, videók kliensoldali lejátszásával HLS-protokollon továbbítva, valamint az NGINX webszerver RTMP-moduljának beállításával, amely lehetővé teszi a videók élő közvetítését.

Az elkészült rendszerből a diplomatervezés során alakítottam ki egy natív AWS-felhőalapú szoftverrendszert, némely komponens újrafelhasználásával előzőből -- mint például a React-alapú kliensoldali weboldal, a Docker-konfiguráció, illetve a szerveroldali kód CRUD-funkciói.
