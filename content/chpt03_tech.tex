\chapter{Felhasznált technológiák}

A fejezet célja, hogy bemutassa a videó streaming szolgáltatások implementációjához felhasznált konkrét szoftvereket, webes és felhőszolgáltatásokat, és az azok közötti kapcsolatokat.

\section{Az FFMpeg szoftvercsomag}

Az FFMpeg egy nyílt forráskódú és GPL-licenszelésű szoftvercsomag, amely képes videók és hangok kódolására, dekódolására, átalakítására (konvertálás), valamint streamelésre. Az FFMpeg a legtöbb operációs rendszeren elérhető, és számos különböző formátumot, modern kodekeket támogat. Az FFMpeg a videók és hangok kódolására és dekódolására szolgáló kodekeket tartalmazza, valamint számos különböző formátumot támogat, beleértve az MPEG-4, H.264, H.265, VP8, VP9, AV1, AAC, AC-3, Opus, és sok más formátumot. Folyamatosan frissen tartják a kodekeket, aktív fejlesztőbázissal rendelkezik.

A lokális videókonvertálásra és streamelésre az FFMpeg-csomagból az azonos nevű ffmpeg parancssori interfészt használtam. Beépítettem a webszerver alkalmazásba egy külön szolgáltatásrészt, amely kihív a webszerver folyamatából és egy megfelelően felparaméterezett ffmpeg parancsot futtat a videókonvertálásra és streamelés megkezdésére aszinkron módon, a kimeneti fájlokat a megfelelő helyre menti.

A felhő alapú megoldásban már nem került felhasználásra az FFMpeg, mivel az AWS Elemental szolgáltatásokat használtam a videókódolásra és streamelésre, viszont a szoftvercsomag közvetlen megismerése a videókonvertálás folyamatának megértését és a konvertálási folyamatok felparaméterezési lehetőségeinek mélyebb átlátását segítette.

\section{Amazon Web Services}

Az Amazon Web Services (AWS) a világ egyik legjobban elterjedt, legnagyobb szerverfarmjait fenntartó, nagy hírű vállalatok által is megbízható felhőszolgáltatója. Felhasználói számára számítási, hálózati, adattárolási célokat megvalósító szolgáltatások széles palettáját kínálja. Felhasználják az AWS-t a mesterséges intelligencia területén; valamint kiterjedt adatbázisok, adatfeldolgozó rendszerek építésére; megbízható és könnyen skálázható webes szoftverrendszerek kialakítására.

A felhasználók a szolgáltatásokhoz az AWS Management Console webes felületen keresztül, vagy az AWS Command Line Interface (AWS CLI) parancssori interfészén keresztül férhetnek hozzá. Az egyes felhasználók fel tudnak állítani maguknak egy vagy több AWS fiókot, amelyek a számlázás és a jogosultságkezelés szempontjából elkülönülhetnek egymástól.

A fiókon belül lehetőséget kapunk granuláris jogosultságkezelésre, azaz az egyes felhasználók, szolgáltatások, vagy szolgáltatásrészek számára különböző alacsony szintű jogosultságokat adhatunk meg.

Az AWS regionális adatközpontokat üzemeltet a világ számos pontján, amelyek közül a felhasználók választhatnak, hogy melyik adatközpontban szeretnének szolgáltatásokat futtatni.

A költségeket ``pay-as-you-go'' alapelv alapján számolják fel, azaz a felhasználók csak az általuk használt szolgáltatások számítási kapacitásáért, a tárhelyért, az adatközpontból kifelé történő hálózati forgalomért fizetnek.

\subsection{AWS Elemental}

TODO: MediaConvert

TODO: MediaLive

TODO: MediaPackage

\subsection{Amazon IVS}

TODO: Az Amazon Interactive Video Service (Amazon IVS) 

TODO: kép innen https://aws.amazon.com/blogs/media/awse-choosing-aws-live-streaming-solution-for-use-case/

Ezen szolgáltatás ismertetése a megértést és a választás indoklását szolgálja későbbi fejezetekben, az Amazon IVS nem került felhasználásra a konkrét lefejlesztett rendszerben.

\subsection{Amazon VPC}

Az Amazon Virtual Private Cloud (Amazon VPC) egy virtuális hálózati környezet. Benne megvalósítható, hogy az AWS publikus felhőjén belül is privát és publikus saját hálózatokat hozzunk létre. Az Amazon VPC segítségével a felhasználók teljes kontrollt gyakorolhatnak a virtuális hálózati környezetük felett, beleértve a VPC-k alatti alhálózatok IP-tartományainak konfigurálását, útvonaltáblák kitöltését, a hálózati interfészek/portok korlátozásait, egyéb hálózati eszközök beillesztését (NAT Gateway-ek, Internet Gateway-ek, valamint Transit Gateway-ek), felülvizsgálható benne a hálózati teljesítmény.

\subsection{Amazon ALB}

Az Application Load Balancer (ALB) az Amazon Elastic Load Balancer (ELB) egy fajtája, ISO-OSI Layer 7 szinten, azaz alkalmazásrétegek szintjén működő terheléselosztó hálózati eszköz. Automatikusan skálázódik, lehetővé teszi a felhasználók számára, hogy egy vagy több szerver példány között egyenletesen eloszthassák a beérkező HTTP- és HTTPS-kéréseket. Az ALB képes a kéréseket a kérések fejlécei alapján vagy a kérések útvonala alapján elkülöníteni a forgalmat.

\subsection{Amazon S3}

Az Amazon Simple Storage Service (Amazon S3) egy objektumtároló szolgáltatás, amely lehetővé teszi a felhasználók számára, hogy nagy mennyiségű adatot tároljanak az AWS-felhőben ``vödrökben'' (bucketokban). Az objektum egy fájl és a fájlt leíró metaadatok közösen. A vödör az objektumok tárolója.

\subsection{Amazon CloudFront}

Az előző fejezet egy szekciójában már említetésre kerültek a CDN mint a video-on-demand alapú streaming szolgáltatások egyik kulcsfontosságú eleme. Az Amazon CloudFront egy globális CDN, amely lehetővé teszi a felhasználók számára, hogy a tartalmat hozzájuk közelebbi szervereken tárolt cache-ből töltsék le, ezáltal csökkentve a késleltetést, csökkentve a központi szerverek terhelését, és növelve a letöltési sebességet. Tartalmaink csoportosítására CloudFront ``disztribúciókat'' használunk.

A disztribúciók különböző URI-útvonalaikon akár különböző CDN-forrásokból -- úgynevezett ``originekből'' -- tudnak tartalmat kiszolgálni: ilyen origin lehet egy S3 vödör, AWS Elemental MediaPackage alapú élőadás-csatorna, Amazon Application Load Balancer (ALB) példány, vagy akár egy egyéni HTTP-szerver is saját doménnévvel.

\subsection{Amazon ECS}

Az Amazon Elastic Container Service (ECS) arra szolgál, hogy konténer alapú alkalmazásokat, szoftvercsomagokat futtathassunk a felhőszolgáltatónál. Az ECS segítségével a felhasználók könnyen futtathatnak és skálázhatnak konténereket anélkül, hogy a konténerek futtatásához szükséges infrastruktúra mélyén futó szervergépeket, valamint azok életciklusát, operációs rendszerének patchelését kellene kezelniük -- ezek menedzselését az AWS Fargate motor veszi át, mi csupán a környezeti paramétereket kell felkonfiguráljuk az igényeinknek megfelelően.

Ilyen paraméterek a konténerek képei, a konténerek alapvető számítási erőforrásai (CPU-magok száma, memória mérete), a konténerek hálózati beállításai (porttovábbítások, alkalmazott Security Group), a konténerek naplózása (hova továbbítódjanak a futtatás során a naplók), és a konténerek hozzáférési jogosultságai az AWS felhőn belüli más szolgáltatásokhoz. Könnyedén kapcsolható össze Amazon ALB példánnyal.

Tipikusan alkalmazott az ECS párban az Amazon Elastic Container Registry (ECR) szolgáltatással, amely egy konténerképek tárolására szolgáló privát Docker Registry, amely lehetővé teszi a felhasználók számára, hogy a konténerek képeit biztonságosan tárolják és kezeljék az AWS felhőben.

\subsection{Amazon RDS}

Az Amazon Relational Database Service (RDS) egy relációs adatbázis szolgáltatás, amely segít, hogy könnyen és hatékonyan hozhassunk létre, üzemeltessünk és skálázzunk relációs adatbázisokat az AWS-felhőben. Az RDS támogatja a legnépszerűbb relációs adatbázis motorokat, mint például a PostgreSQL, MySQL, MariaDB, Oracle, és SQL Server. Képes automatikusan kezelni az adatbázisok frissítéseinek telepítését és a folyamatos biztonsági mentéseket.

\subsection{Kiegészítő AWS szolgáltatások}

A konténerek orkesztrációjának kiegészítésére számos könnyen élesíthető és ECS-hez integrálható szolgáltatás áll rendelkezésre az AWS-felhőben, amelyek közül a legelterjedtebbek az Amazon CloudWatch Logs, az AWS Lambda és a Amazon EventBridge.

Az Amazon CloudWatch Logs egy naplózó és monitorozó szolgáltatás, amely lehetővé teszi a felhasználók számára, hogy a konténerek futtatása során keletkező naplókat gyűjtsék, tárolják, és vizsgálják az ezekből származó metrikákat is akár.

Az AWS Lambda egy serverless Function-as-a-Service (FaaS) szolgáltatás, amely lehetővé teszi kód függvényszerű futtatását anélkül, hogy szükség lenne a szerverek vagy a futtatási környezet menedzselésére. A Lambda-függvény eseményekre reagálva kerül meghívásra, például HTTP kérésekre, adatbázis eseményekre, vagy más AWS-szolgáltatások eseményeire.

Ezzel kapcsolatban kerül a képbe az EventBridge, az AWS központi eseménykezelő szolgáltatása, amely lehetővé teszi az egyes AWS-felhőszolgátatásokon futó alrendszerek közötti kommunikációt. Segítségével szűrhetünk eseményekre, azokat könnyen továbbíthatjuk az egyes AWS-szolgáltatások között, a célpontja egy EventBridge által elkapott eseménynek ennek megfelelően egy Lambda-függvény is lehet.

\section{A webes komponensek technológiái}

A különböző felhőszolgáltatásokon futó kódbázisokat elterjedt webes technológiák segítségével fejlesztettem. Ezen technológiák kerülnek bemutatásra a következő szekciókban.

\subsection{TypeScript és JavaScript nyelvek}

TODO: Miért választottam a TypeScriptet, miért jobb a JavaScriptnél, felhasználása a JavaScriptnek a Lambda és Cloudfront Function-ökben.

\subsection{Node.js ökoszisztéma}

TODO: Node.js futtatókörnyezet, NestJS keretrendszer, Prisma ORM, AWS SDK.

\subsection{React}

TODO: Mire jó a React: Statikus oldalak generálása (SSG), egyszerűbb JavaScript egy helyen kezelése, gyors build toolok, erre ott a Vite.js.

TODO: Sok könnyen integrálható könyvtár, amelyek segítségével gyorsan és hatékonyan lehet webes felületeket fejleszteni: React Hook Forms, React Query.

\subsection{Docker}

TODO

\subsection{GitHub}

Szoftverrendszerek, alkalmazások fejlesztése során szinte elengedhetetlen a verziókezelés, amelynek segítségével a fejlesztők nyomonkövethetik a kódbázis változásait, visszaállíthatják az előző verziókat, és könnyen együtt tudnak dolgozni a kódon.

Erre a munkafolyamatra az egyik legelterjedtebb Source Code Management (SCM) eszköz a Git verziókezelő. A Git egy elosztott verziókezelő rendszer. Minden fejlesztő saját gépén tárolja a teljes kódbázisát, majd a módosításokat a felhőben lévő tárolóval szinkronizálhatja.

Az elterjedt Git felhőszolgáltatók közül a GitHub platformot választottam, és a GitHub Actions CI/CD szolgáltatását használtam a kódminőség ellenőrzésére, a kódépítés automatizálására, és a kód AWS-re való kiélesítésére is.
